% BSF Guidelines
%Please upload a short abstract with the name of the project in lay terms, the names and affiliations of the PIs, and a short description of the project, also in lay terms. Be sure to include the reasoning behind the proposed research, its significance and its impact, if successful, on humanity, the environment or on the scientific field. This abstract must be written in a way that non-scientists will understand the main facts.


%%%%%%%%%%% BSF GUIDELINES FOR ABSTRACT FILE %%%%%%%%%%%%
%This file should include the word 'Abstract' at the top. The following information should be included: the full title of the proposed application, which should be brief, meaningful and suitable for use in the general media; the application number supplied by the system; and the names and affiliations of the principal investigators.
%
%An abstract of the proposed research of 250 words or less is required. If a grant is awarded, the abstract may be sent to science information exchange centers and become available to the public. The abstract should be informative to scientists in the same or related fields. A statement of the project's potential contribution to the research done in that field should be included.
%%%%%%%%%%%%%%%%%%%%%%%%%%%%%%%%%%%%%%%%%%%%%%%%%%%%%%%%%


\documentclass[12pt]{article}

\usepackage[ruled,vlined,linesnumbered]{algorithm2e}

\usepackage{times}
\usepackage{anysize}
\marginsize{2cm}{2cm}{2cm}{3cm}
%\onehalfspace
%\doublespace
\setlength{\parindent}{0.8cm}
%\setlength{\parskip}{0.2\baselineskip}
%\setlength{\topmargin}{2cm}
%\setlength{\textheight}{25cm}
%\setlength{\textwidth}{14cm}
%\setlength{\oddsidemargin}{2cm}
%\setlength{\evensidemargin}{2cm}
\usepackage{fancyhdr}
\pagestyle{fancy}


\linespread{1.3}

\lhead{Data-and-Model Driven Reasoning}
\cfoot{\thepage} 
%\cfoot{} 
\pagenumbering{arabic}
%\pagenumbering{Roman}

\begin{document}


\title{Short Abstract in Lay Terms \\ \Large{Data-and-Model Driven Reasoning, BSF Application No. 2016141}}
\date{\vspace{-0.5cm}}
\author{Roni Stern \\ Software and Information Systems Eng., Ben Gurion University of the Negev
        \and Brendan Juba \\ Computer Science, Washington University in St. Louis}
\maketitle

% There are two main approached: model-based and data-drive 
Research in Artificial Intelligence (AI) is torn between two general approaches: {\em model-based} and {\em data-driven}. 
Traditional AI approaches are {\em model-based}, which means that some model that describes how the world is expected to behave is given, e.g., the specification of a physical system, and is used to reason about past and future events. By contrast, {\em data-driven} methods do not assume that such a model is given, but do assume that observations about the world are collected, and are used to predict what is expected to occur in the future. 


In the proposed research we focus on two important AI tasks -- autonomous planning and automated diagnosis -- and explore ways in which the two general approaches -- model-based and data-driven -- 
can be integrated in a synergistic manner. The combination of these approaches enjoys their complementary benefits and leads the way towards unifying these two fundamental approaches. 
%for AI, and propose theory and algorithms that use both models of the world as well as past observations. Specifically, we will focus on model-and-observations algorithms for two classical AI tasks: autonomous planning and automated diagnosis. 


\end{document}

