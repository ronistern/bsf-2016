%%%%%%%%%%% BSF GUIDELINES FOR ABSTRACT FILE %%%%%%%%%%%%
%This file should include the word 'Abstract' at the top. The following information should be included: the full title of the proposed application, which should be brief, meaningful and suitable for use in the general media; the application number supplied by the system; and the names and affiliations of the principal investigators.
%
%An abstract of the proposed research of 250 words or less is required. If a grant is awarded, the abstract may be sent to science information exchange centers and become available to the public. The abstract should be informative to scientists in the same or related fields. A statement of the project's potential contribution to the research done in that field should be included.
%%%%%%%%%%%%%%%%%%%%%%%%%%%%%%%%%%%%%%%%%%%%%%%%%%%%%%%%%


\documentclass[12pt]{article}

\usepackage[ruled,vlined,linesnumbered]{algorithm2e}

\usepackage{times}
\usepackage{anysize}
\marginsize{2cm}{2cm}{2cm}{3cm}
%\onehalfspace
%\doublespace
\setlength{\parindent}{0.8cm}
%\setlength{\parskip}{0.2\baselineskip}
%\setlength{\topmargin}{2cm}
%\setlength{\textheight}{25cm}
%\setlength{\textwidth}{14cm}
%\setlength{\oddsidemargin}{2cm}
%\setlength{\evensidemargin}{2cm}
\usepackage{fancyhdr}
\pagestyle{fancy}


\linespread{1.3}

\lhead{Data-and-Model Driven Reasoning} \rhead{R. Stern and B. Juba}
\cfoot{\thepage} 
%\cfoot{} 
\pagenumbering{arabic}
%\pagenumbering{Roman}

\begin{document}


\title{Abstract \\ \Large{Data-and-Model Driven Reasoning, BSF Application No. 2016141}}
\date{\vspace{-0.5cm}}
\author{Roni Stern \\ Software and Information Systems Eng., Ben Gurion University of the Negev
        \and Brendan Juba \\ Computer Science, Washington University in St. Louis}
\maketitle

% There are two main approached: model-based and data-drive 
Research in Artificial Intelligence is torn between two general approaches: the traditional AI approach that assume a model of the world is given and {\bf reasons} about it, and the data-driven  approach that builds on the growing availability of data and {\bf learns} from it how to handle future events. For example, classical planning algorithms 
employ a model-based approach through the well-known ``frame axioms'' assumed by STRIPS planners. In contrast, popular reinforcement learning algorithms do not even try to model the environment, and learn directly how to act. Both approaches -- model-based and data-driven -- have pros and cons: model-based methods are used when a reliable model of the world is given while 
data-driven methods are usually used when observations of past activities are given instead of such a model. % and model-based methods are used when a reliable model of the world is given and are usually applied suited for different settings. 
%In general, one can say that the model-based approaches are often more principled and allow stronger claims about the performance of the underlying algorithms (e.g., optimality). However, as countless applications have discovered, having an accurate model of the world is very difficult. These are not needed by data-driven methods, which are also often much faster than the model-based methods. 



% Data driven is not always great, and model-based is not always great. Here lays the challenge
%Clearly, if there is no a-priori knowledge about the world model then data-driven methods are needed. On the other hand, if an accurate model of the world is available then it is wasteful to ignore it. Moreover, model-based approach provide strong theoretical guarantees in terms of completeness and solution quality (e.g., optimality) that can be difficult, if not impossible, to provide without a formal model of the world. 


% We aim for the middle ground
% Our goal: when to use which and why
In the proposed research we will explore how model-based and data-driven approaches can augment each other in the broad range of settings where both model and observations are available. We explore these model-and-observation settings for two important and well-studied problems: automated planning and automated diagnosis. 
Our objective is to developed planning algorithms and diagnosis algorithms that enjoy the complimentary benefits of the model-based and the data-driven aproaches, and use both sources of information in a principled way. 


\end{document}

